
%% bare_adv.tex
%% V1.4b
%% 2015/08/26
%% by Michael Shell
%% See: 
%% http://www.michaelshell.org/
%% for current contact information.
%%
%% This is a skeleton file demonstrating the advanced use of IEEEtran.cls
%% (requires IEEEtran.cls version 1.8b or later) with an IEEE Computer
%% Society journal paper.
%%
%% Support sites:
%% http://www.michaelshell.org/tex/ieeetran/
%% http://www.ctan.org/pkg/ieeetran
%% and
%% http://www.ieee.org/

%%*************************************************************************
%% Legal Notice:
%% This code is offered as-is without any warranty either expressed or
%% implied; without even the implied warranty of MERCHANTABILITY or
%% FITNESS FOR A PARTICULAR PURPOSE! 
%% User assumes all risk.
%% In no event shall the IEEE or any contributor to this code be liable for
%% any damages or losses, including, but not limited to, incidental,
%% consequential, or any other damages, resulting from the use or misuse
%% of any information contained here.
%%
%% All comments are the opinions of their respective authors and are not
%% necessarily endorsed by the IEEE.
%%
%% This work is distributed under the LaTeX Project Public License (LPPL)
%% ( http://www.latex-project.org/ ) version 1.3, and may be freely used,
%% distributed and modified. A copy of the LPPL, version 1.3, is included
%% in the base LaTeX documentation of all distributions of LaTeX released
%% 2003/12/01 or later.
%% Retain all contribution notices and credits.
%% ** Modified files should be clearly indicated as such, including  **
%% ** renaming them and changing author support contact information. **
%%*************************************************************************


% *** Authors should verify (and, if needed, correct) their LaTeX system  ***
% *** with the testflow diagnostic prior to trusting their LaTeX platform ***
% *** with production work. The IEEE's font choices and paper sizes can   ***
% *** trigger bugs that do not appear when using other class files.       ***                          ***
% The testflow support page is at:
% http://www.michaelshell.org/tex/testflow/


% IEEEtran V1.7 and later provides for these CLASSINPUT macros to allow the
% user to reprogram some IEEEtran.cls defaults if needed. These settings
% override the internal defaults of IEEEtran.cls regardless of which class
% options are used. Do not use these unless you have good reason to do so as
% they can result in nonIEEE compliant documents. User beware. ;)
%
%\newcommand{\CLASSINPUTbaselinestretch}{1.0} % baselinestretch
%\newcommand{\CLASSINPUTinnersidemargin}{1in} % inner side margin
%\newcommand{\CLASSINPUToutersidemargin}{1in} % outer side margin
%\newcommand{\CLASSINPUTtoptextmargin}{1in}   % top text margin
%\newcommand{\CLASSINPUTbottomtextmargin}{1in}% bottom text margin




%
\documentclass[10pt,journal,compsoc]{IEEEtran}
% If IEEEtran.cls has not been installed into the LaTeX system files,
% manually specify the path to it like:
% \documentclass[10pt,journal,compsoc]{../sty/IEEEtran}


% For Computer Society journals, IEEEtran defaults to the use of 
% Palatino/Palladio as is done in IEEE Computer Society journals.
% To go back to Times Roman, you can use this code:
%\renewcommand{\rmdefault}{ptm}\selectfont





% Some very useful LaTeX packages include:
% (uncomment the ones you want to load)
\usepackage{graphicx}
\usepackage{hyperref}


% *** MISC UTILITY PACKAGES ***
%
%\usepackage{ifpdf}
% Heiko Oberdiek's ifpdf.sty is very useful if you need conditional
% compilation based on whether the output is pdf or dvi.
% usage:
% \ifpdf
%   % pdf code
% \else
%   % dvi code
% \fi
% The latest version of ifpdf.sty can be obtained from:
% http://www.ctan.org/pkg/ifpdf
% Also, note that IEEEtran.cls V1.7 and later provides a builtin
% \ifCLASSINFOpdf conditional that works the same way.
% When switching from latex to pdflatex and vice-versa, the compiler may
% have to be run twice to clear warning/error messages.






% *** CITATION PACKAGES ***
%
\ifCLASSOPTIONcompsoc
  % The IEEE Computer Society needs nocompress option
  % requires cite.sty v4.0 or later (November 2003)
  \usepackage[nocompress]{cite}
\else
  % normal IEEE
  \usepackage{cite}
\fi
% cite.sty was written by Donald Arseneau
% V1.6 and later of IEEEtran pre-defines the format of the cite.sty package
% \cite{} output to follow that of the IEEE. Loading the cite package will
% result in citation numbers being automatically sorted and properly
% "compressed/ranged". e.g., [1], [9], [2], [7], [5], [6] without using
% cite.sty will become [1], [2], [5]--[7], [9] using cite.sty. cite.sty's
% \cite will automatically add leading space, if needed. Use cite.sty's
% noadjust option (cite.sty V3.8 and later) if you want to turn this off
% such as if a citation ever needs to be enclosed in parenthesis.
% cite.sty is already installed on most LaTeX systems. Be sure and use
% version 5.0 (2009-03-20) and later if using hyperref.sty.
% The latest version can be obtained at:
% http://www.ctan.org/pkg/cite
% The documentation is contained in the cite.sty file itself.
%
% Note that some packages require special options to format as the Computer
% Society requires. In particular, Computer Society  papers do not use
% compressed citation ranges as is done in typical IEEE papers
% (e.g., [1]-[4]). Instead, they list every citation separately in order
% (e.g., [1], [2], [3], [4]). To get the latter we need to load the cite
% package with the nocompress option which is supported by cite.sty v4.0
% and later.





% *** GRAPHICS RELATED PACKAGES ***
%
\ifCLASSINFOpdf
  % \usepackage[pdftex]{graphicx}
  % declare the path(s) where your graphic files are
  % \graphicspath{{../pdf/}{../jpeg/}}
  % and their extensions so you won't have to specify these with
  % every instance of \includegraphics
  % \DeclareGraphicsExtensions{.pdf,.jpeg,.png}
\else
  % or other class option (dvipsone, dvipdf, if not using dvips). graphicx
  % will default to the driver specified in the system graphics.cfg if no
  % driver is specified.
  % \usepackage[dvips]{graphicx}
  % declare the path(s) where your graphic files are
  % \graphicspath{{../eps/}}
  % and their extensions so you won't have to specify these with
  % every instance of \includegraphics
  % \DeclareGraphicsExtensions{.eps}
\fi
% graphicx was written by David Carlisle and Sebastian Rahtz. It is
% required if you want graphics, photos, etc. graphicx.sty is already
% installed on most LaTeX systems. The latest version and documentation
% can be obtained at: 
% http://www.ctan.org/pkg/graphicx
% Another good source of documentation is "Using Imported Graphics in
% LaTeX2e" by Keith Reckdahl which can be found at:
% http://www.ctan.org/pkg/epslatex
%
% latex, and pdflatex in dvi mode, support graphics in encapsulated
% postscript (.eps) format. pdflatex in pdf mode supports graphics
% in .pdf, .jpeg, .png and .mps (metapost) formats. Users should ensure
% that all non-photo figures use a vector format (.eps, .pdf, .mps) and
% not a bitmapped formats (.jpeg, .png). The IEEE frowns on bitmapped formats
% which can result in "jaggedy"/blurry rendering of lines and letters as
% well as large increases in file sizes.
%
% You can find documentation about the pdfTeX application at:
% http://www.tug.org/applications/pdftex





% *** MATH PACKAGES ***
%
%\usepackage{amsmath}
% A popular package from the American Mathematical Society that provides
% many useful and powerful commands for dealing with mathematics.
%
% Note that the amsmath package sets \interdisplaylinepenalty to 10000
% thus preventing page breaks from occurring within multiline equations. Use:
%\interdisplaylinepenalty=2500
% after loading amsmath to restore such page breaks as IEEEtran.cls normally
% does. amsmath.sty is already installed on most LaTeX systems. The latest
% version and documentation can be obtained at:
% http://www.ctan.org/pkg/amsmath





% *** SPECIALIZED LIST PACKAGES ***
%\usepackage{acronym}
% acronym.sty was written by Tobias Oetiker. This package provides tools for
% managing documents with large numbers of acronyms. (You don't *have* to
% use this package - unless you have a lot of acronyms, you may feel that
% such package management of them is bit of an overkill.)
% Do note that the acronym environment (which lists acronyms) will have a
% problem when used under IEEEtran.cls because acronym.sty relies on the
% description list environment - which IEEEtran.cls has customized for
% producing IEEE style lists. A workaround is to declared the longest
% label width via the IEEEtran.cls \IEEEiedlistdecl global control:
%
% \renewcommand{\IEEEiedlistdecl}{\IEEEsetlabelwidth{SONET}}
% \begin{acronym}
%
% \end{acronym}
% \renewcommand{\IEEEiedlistdecl}{\relax}% remember to reset \IEEEiedlistdecl
%
% instead of using the acronym environment's optional argument.
% The latest version and documentation can be obtained at:
% http://www.ctan.org/pkg/acronym


%\usepackage{algorithmic}
% algorithmic.sty was written by Peter Williams and Rogerio Brito.
% This package provides an algorithmic environment fo describing algorithms.
% You can use the algorithmic environment in-text or within a figure
% environment to provide for a floating algorithm. Do NOT use the algorithm
% floating environment provided by algorithm.sty (by the same authors) or
% algorithm2e.sty (by Christophe Fiorio) as the IEEE does not use dedicated
% algorithm float types and packages that provide these will not provide
% correct IEEE style captions. The latest version and documentation of
% algorithmic.sty can be obtained at:
% http://www.ctan.org/pkg/algorithms
% Also of interest may be the (relatively newer and more customizable)
% algorithmicx.sty package by Szasz Janos:
% http://www.ctan.org/pkg/algorithmicx




% *** ALIGNMENT PACKAGES ***
%
%\usepackage{array}
% Frank Mittelbach's and David Carlisle's array.sty patches and improves
% the standard LaTeX2e array and tabular environments to provide better
% appearance and additional user controls. As the default LaTeX2e table
% generation code is lacking to the point of almost being broken with
% respect to the quality of the end results, all users are strongly
% advised to use an enhanced (at the very least that provided by array.sty)
% set of table tools. array.sty is already installed on most systems. The
% latest version and documentation can be obtained at:
% http://www.ctan.org/pkg/array


%\usepackage{mdwmath}
%\usepackage{mdwtab}
% Also highly recommended is Mark Wooding's extremely powerful MDW tools,
% especially mdwmath.sty and mdwtab.sty which are used to format equations
% and tables, respectively. The MDWtools set is already installed on most
% LaTeX systems. The lastest version and documentation is available at:
% http://www.ctan.org/pkg/mdwtools


% IEEEtran contains the IEEEeqnarray family of commands that can be used to
% generate multiline equations as well as matrices, tables, etc., of high
% quality.


%\usepackage{eqparbox}
% Also of notable interest is Scott Pakin's eqparbox package for creating
% (automatically sized) equal width boxes - aka "natural width parboxes".
% Available at:
% http://www.ctan.org/pkg/eqparbox




% *** SUBFIGURE PACKAGES ***
%\ifCLASSOPTIONcompsoc
%  \usepackage[caption=false,font=footnotesize,labelfont=sf,textfont=sf]{subfig}
%\else
%  \usepackage[caption=false,font=footnotesize]{subfig}
%\fi
% subfig.sty, written by Steven Douglas Cochran, is the modern replacement
% for subfigure.sty, the latter of which is no longer maintained and is
% incompatible with some LaTeX packages including fixltx2e. However,
% subfig.sty requires and automatically loads Axel Sommerfeldt's caption.sty
% which will override IEEEtran.cls' handling of captions and this will result
% in non-IEEE style figure/table captions. To prevent this problem, be sure
% and invoke subfig.sty's "caption=false" package option (available since
% subfig.sty version 1.3, 2005/06/28) as this is will preserve IEEEtran.cls
% handling of captions.
% Note that the Computer Society format requires a sans serif font rather
% than the serif font used in traditional IEEE formatting and thus the need
% to invoke different subfig.sty package options depending on whether
% compsoc mode has been enabled.
%
% The latest version and documentation of subfig.sty can be obtained at:
% http://www.ctan.org/pkg/subfig




% *** FLOAT PACKAGES ***
%
%\usepackage{fixltx2e}
% fixltx2e, the successor to the earlier fix2col.sty, was written by
% Frank Mittelbach and David Carlisle. This package corrects a few problems
% in the LaTeX2e kernel, the most notable of which is that in current
% LaTeX2e releases, the ordering of single and double column floats is not
% guaranteed to be preserved. Thus, an unpatched LaTeX2e can allow a
% single column figure to be placed prior to an earlier double column
% figure.
% Be aware that LaTeX2e kernels dated 2015 and later have fixltx2e.sty's
% corrections already built into the system in which case a warning will
% be issued if an attempt is made to load fixltx2e.sty as it is no longer
% needed.
% The latest version and documentation can be found at:
% http://www.ctan.org/pkg/fixltx2e


%\usepackage{stfloats}
% stfloats.sty was written by Sigitas Tolusis. This package gives LaTeX2e
% the ability to do double column floats at the bottom of the page as well
% as the top. (e.g., "\begin{figure*}[!b]" is not normally possible in
% LaTeX2e). It also provides a command:
%\fnbelowfloat
% to enable the placement of footnotes below bottom floats (the standard
% LaTeX2e kernel puts them above bottom floats). This is an invasive package
% which rewrites many portions of the LaTeX2e float routines. It may not work
% with other packages that modify the LaTeX2e float routines. The latest
% version and documentation can be obtained at:
% http://www.ctan.org/pkg/stfloats
% Do not use the stfloats baselinefloat ability as the IEEE does not allow
% \baselineskip to stretch. Authors submitting work to the IEEE should note
% that the IEEE rarely uses double column equations and that authors should try
% to avoid such use. Do not be tempted to use the cuted.sty or midfloat.sty
% packages (also by Sigitas Tolusis) as the IEEE does not format its papers in
% such ways.
% Do not attempt to use stfloats with fixltx2e as they are incompatible.
% Instead, use Morten Hogholm'a dblfloatfix which combines the features
% of both fixltx2e and stfloats:
%
% \usepackage{dblfloatfix}
% The latest version can be found at:
% http://www.ctan.org/pkg/dblfloatfix


%\ifCLASSOPTIONcaptionsoff
%  \usepackage[nomarkers]{endfloat}
% \let\MYoriglatexcaption\caption
% \renewcommand{\caption}[2][\relax]{\MYoriglatexcaption[#2]{#2}}
%\fi
% endfloat.sty was written by James Darrell McCauley, Jeff Goldberg and 
% Axel Sommerfeldt. This package may be useful when used in conjunction with 
% IEEEtran.cls'  captionsoff option. Some IEEE journals/societies require that
% submissions have lists of figures/tables at the end of the paper and that
% figures/tables without any captions are placed on a page by themselves at
% the end of the document. If needed, the draftcls IEEEtran class option or
% \CLASSINPUTbaselinestretch interface can be used to increase the line
% spacing as well. Be sure and use the nomarkers option of endfloat to
% prevent endfloat from "marking" where the figures would have been placed
% in the text. The two hack lines of code above are a slight modification of
% that suggested by in the endfloat docs (section 8.4.1) to ensure that
% the full captions always appear in the list of figures/tables - even if
% the user used the short optional argument of \caption[]{}.
% IEEE papers do not typically make use of \caption[]'s optional argument,
% so this should not be an issue. A similar trick can be used to disable
% captions of packages such as subfig.sty that lack options to turn off
% the subcaptions:
% For subfig.sty:
% \let\MYorigsubfloat\subfloat
% \renewcommand{\subfloat}[2][\relax]{\MYorigsubfloat[]{#2}}
% However, the above trick will not work if both optional arguments of
% the \subfloat command are used. Furthermore, there needs to be a
% description of each subfigure *somewhere* and endfloat does not add
% subfigure captions to its list of figures. Thus, the best approach is to
% avoid the use of subfigure captions (many IEEE journals avoid them anyway)
% and instead reference/explain all the subfigures within the main caption.
% The latest version of endfloat.sty and its documentation can obtained at:
% http://www.ctan.org/pkg/endfloat
%
% The IEEEtran \ifCLASSOPTIONcaptionsoff conditional can also be used
% later in the document, say, to conditionally put the References on a 
% page by themselves.





% *** PDF, URL AND HYPERLINK PACKAGES ***
%
%\usepackage{url}
% url.sty was written by Donald Arseneau. It provides better support for
% handling and breaking URLs. url.sty is already installed on most LaTeX
% systems. The latest version and documentation can be obtained at:
% http://www.ctan.org/pkg/url
% Basically, \url{my_url_here}.


% NOTE: PDF thumbnail features are not required in IEEE papers
%       and their use requires extra complexity and work.
%\ifCLASSINFOpdf
%  \usepackage[pdftex]{thumbpdf}
%\else
%  \usepackage[dvips]{thumbpdf}
%\fi
% thumbpdf.sty and its companion Perl utility were written by Heiko Oberdiek.
% It allows the user a way to produce PDF documents that contain fancy
% thumbnail images of each of the pages (which tools like acrobat reader can
% utilize). This is possible even when using dvi->ps->pdf workflow if the
% correct thumbpdf driver options are used. thumbpdf.sty incorporates the
% file containing the PDF thumbnail information (filename.tpm is used with
% dvips, filename.tpt is used with pdftex, where filename is the base name of
% your tex document) into the final ps or pdf output document. An external
% utility, the thumbpdf *Perl script* is needed to make these .tpm or .tpt
% thumbnail files from a .ps or .pdf version of the document (which obviously
% does not yet contain pdf thumbnails). Thus, one does a:
% 
% thumbpdf filename.pdf 
%
% to make a filename.tpt, and:
%
% thumbpdf --mode dvips filename.ps
%
% to make a filename.tpm which will then be loaded into the document by
% thumbpdf.sty the NEXT time the document is compiled (by pdflatex or
% latex->dvips->ps2pdf). Users must be careful to regenerate the .tpt and/or
% .tpm files if the main document changes and then to recompile the
% document to incorporate the revised thumbnails to ensure that thumbnails
% match the actual pages. It is easy to forget to do this!
% 
% Unix systems come with a Perl interpreter. However, MS Windows users
% will usually have to install a Perl interpreter so that the thumbpdf
% script can be run. The Ghostscript PS/PDF interpreter is also required.
% See the thumbpdf docs for details. The latest version and documentation
% can be obtained at.
% http://www.ctan.org/pkg/thumbpdf


% NOTE: PDF hyperlink and bookmark features are not required in IEEE
%       papers and their use requires extra complexity and work.
% *** IF USING HYPERREF BE SURE AND CHANGE THE EXAMPLE PDF ***
% *** TITLE/SUBJECT/AUTHOR/KEYWORDS INFO BELOW!!           ***
\newcommand\MYhyperrefoptions{bookmarks=true,bookmarksnumbered=true,
pdfpagemode={UseOutlines},plainpages=false,pdfpagelabels=true,
colorlinks=true,linkcolor={black},citecolor={black},urlcolor={black},
pdftitle={Musical Affective Recommender System (MARS)},%<!CHANGE!
pdfsubject={Sentiment analysis research, music recommendation system report},%<!CHANGE!
pdfauthor={Jesse de Ruijter},%<!CHANGE!
pdfkeywords={Computing science, Music, Sentiment Analysis, Valence Arousal, Million Song Dataset, music recommendation, paper}}%<^!CHANGE!
%\ifCLASSINFOpdf
%\usepackage[\MYhyperrefoptions,pdftex]{hyperref}
%\else
%\usepackage[\MYhyperrefoptions,breaklinks=true,dvips]{hyperref}
%\usepackage{breakurl}
%\fi
% One significant drawback of using hyperref under DVI output is that the
% LaTeX compiler cannot break URLs across lines or pages as can be done
% under pdfLaTeX's PDF output via the hyperref pdftex driver. This is
% probably the single most important capability distinction between the
% DVI and PDF output. Perhaps surprisingly, all the other PDF features
% (PDF bookmarks, thumbnails, etc.) can be preserved in
% .tex->.dvi->.ps->.pdf workflow if the respective packages/scripts are
% loaded/invoked with the correct driver options (dvips, etc.). 
% As most IEEE papers use URLs sparingly (mainly in the references), this
% may not be as big an issue as with other publications.
%
% That said, Vilar Camara Neto created his breakurl.sty package which
% permits hyperref to easily break URLs even in dvi mode.
% Note that breakurl, unlike most other packages, must be loaded
% AFTER hyperref. The latest version of breakurl and its documentation can
% be obtained at:
% http://www.ctan.org/pkg/breakurl
% breakurl.sty is not for use under pdflatex pdf mode.
%
% The advanced features offer by hyperref.sty are not required for IEEE
% submission, so users should weigh these features against the added
% complexity of use.
% The package options above demonstrate how to enable PDF bookmarks
% (a type of table of contents viewable in Acrobat Reader) as well as
% PDF document information (title, subject, author and keywords) that is
% viewable in Acrobat reader's Document_Properties menu. PDF document
% information is also used extensively to automate the cataloging of PDF
% documents. The above set of options ensures that hyperlinks will not be
% colored in the text and thus will not be visible in the printed page,
% but will be active on "mouse over". USING COLORS OR OTHER HIGHLIGHTING
% OF HYPERLINKS CAN RESULT IN DOCUMENT REJECTION BY THE IEEE, especially if
% these appear on the "printed" page. IF IN DOUBT, ASK THE RELEVANT
% SUBMISSION EDITOR. You may need to add the option hypertexnames=false if
% you used duplicate equation numbers, etc., but this should not be needed
% in normal IEEE work.
% The latest version of hyperref and its documentation can be obtained at:
% http://www.ctan.org/pkg/hyperref





% *** Do not adjust lengths that control margins, column widths, etc. ***
% *** Do not use packages that alter fonts (such as pslatex).         ***
% There should be no need to do such things with IEEEtran.cls V1.6 and later.
% (Unless specifically asked to do so by the journal or conference you plan
% to submit to, of course. )


% correct bad hyphenation here
\hyphenation{op-tical net-works semi-conduc-tor}


\begin{document}
%
% paper title
% Titles are generally capitalized except for words such as a, an, and, as,
% at, but, by, for, in, nor, of, on, or, the, to and up, which are usually
% not capitalized unless they are the first or last word of the title.
% Linebreaks \\ can be used within to get better formatting as desired.
% Do not put math or special symbols in the title.
\title{Musical Affective Recommender System (MARS)}
%
%
% author names and IEEE memberships
% note positions of commas and nonbreaking spaces ( ~ ) LaTeX will not break
% a structure at a ~ so this keeps an author's name from being broken across
% two lines.
% use \thanks{} to gain access to the first footnote area
% a separate \thanks must be used for each paragraph as LaTeX2e's \thanks
% was not built to handle multiple paragraphs
%
%
%\IEEEcompsocitemizethanks is a special \thanks that produces the bulleted
% lists the Computer Society journals use for "first footnote" author
% affiliations. Use \IEEEcompsocthanksitem which works much like \item
% for each affiliation group. When not in compsoc mode,
% \IEEEcompsocitemizethanks becomes like \thanks and
% \IEEEcompsocthanksitem becomes a line break with idention. This
% facilitates dual compilation, although admittedly the differences in the
% desired content of \author between the different types of papers makes a
% one-size-fits-all approach a daunting prospect. For instance, compsoc 
% journal papers have the author affiliations above the "Manuscript
% received ..."  text while in non-compsoc journals this is reversed. Sigh.

\author{Armen Shamelian - 3831817, Rens Rooimans - 3862569, Jesse de Ruijter - 3837009, \\Daphne Odekerken - 3827887, Everton Souto Lima - 5595029 \\ Contact: D.Odekerken@uu.nl}% <-this % stops a space
%\thanks{Manuscript received April 19, 2005; revised August 26, 2015.}}

% note the % following the last \IEEEmembership and also \thanks - 
% these prevent an unwanted space from occurring between the last author name
% and the end of the author line. i.e., if you had this:
% 
% \author{....lastname \thanks{...} \thanks{...} }
%                     ^------------^------------^----Do not want these spaces!
%
% a space would be appended to the last name and could cause every name on that
% line to be shifted left slightly. This is one of those "LaTeX things". For
% instance, "\textbf{A} \textbf{B}" will typeset as "A B" not "AB". To get
% "AB" then you have to do: "\textbf{A}\textbf{B}"
% \thanks is no different in this regard, so shield the last } of each \thanks
% that ends a line with a % and do not let a space in before the next \thanks.
% Spaces after \IEEEmembership other than the last one are OK (and needed) as
% you are supposed to have spaces between the names. For what it is worth,
% this is a minor point as most people would not even notice if the said evil
% space somehow managed to creep in.



% The paper headers
%\markboth{Journal of \LaTeX\ Class Files,~Vol.~14, No.~8, August~2015}%
%{Shell \MakeLowercase{\textit{et al.}}: Bare Advanced Demo of IEEEtran.cls for IEEE Computer Society Journals}
% The only time the second header will appear is for the odd numbered pages
% after the title page when using the twoside option.
% 
% *** Note that you probably will NOT want to include the author's ***
% *** name in the headers of peer review papers.                   ***
% You can use \ifCLASSOPTIONpeerreview for conditional compilation here if
% you desire.



% The publisher's ID mark at the bottom of the page is less important with
% Computer Society journal papers as those publications place the marks
% outside of the main text columns and, therefore, unlike regular IEEE
% journals, the available text space is not reduced by their presence.
% If you want to put a publisher's ID mark on the page you can do it like
% this:
%\IEEEpubid{0000--0000/00\$00.00~\copyright~2015 IEEE}
% or like this to get the Computer Society new two part style.
%\IEEEpubid{\makebox[\columnwidth]{\hfill 0000--0000/00/\$00.00~\copyright~2015 IEEE}%
%\hspace{\columnsep}\makebox[\columnwidth]{Published by the IEEE Computer Society\hfill}}
% Remember, if you use this you must call \IEEEpubidadjcol in the second
% column for its text to clear the IEEEpubid mark (Computer Society journal
% papers don't need this extra clearance.)



% use for special paper notices
%\IEEEspecialpapernotice{(Invited Paper)}



% for Computer Society papers, we must declare the abstract and index terms
% PRIOR to the title within the \IEEEtitleabstractindextext IEEEtran
% command as these need to go into the title area created by \maketitle.
% As a general rule, do not put math, special symbols or citations
% in the abstract or keywords.
\IEEEtitleabstractindextext{%
\begin{abstract}
A Music Recommender System is introduced, founded on text-based sentiment analysis and questionnaires to determine the affective load of music signal characteristics, and developed and validated with data from the Million Song Dataset and EchoNest public API. Overall, users of the system found it matched emotions accurately.
\end{abstract}

% Note that keywords are not normally used for peerreview papers.
\begin{IEEEkeywords}
Music, Sentiment Analysis, Valence Arousal, Million Song Dataset
\end{IEEEkeywords}}


% make the title area
\maketitle


% To allow for easy dual compilation without having to reenter the
% abstract/keywords data, the \IEEEtitleabstractindextext text will
% not be used in maketitle, but will appear (i.e., to be "transported")
% here as \IEEEdisplaynontitleabstractindextext when compsoc mode
% is not selected <OR> if conference mode is selected - because compsoc
% conference papers position the abstract like regular (non-compsoc)
% papers do!
\IEEEdisplaynontitleabstractindextext
% \IEEEdisplaynontitleabstractindextext has no effect when using
% compsoc under a non-conference mode.


% For peer review papers, you can put extra information on the cover
% page as needed:
% \ifCLASSOPTIONpeerreview
% \begin{center} \bfseries EDICS Category: 3-BBND \end{center}
% \fi
%
% For peerreview papers, this IEEEtran command inserts a page break and
% creates the second title. It will be ignored for other modes.
\IEEEpeerreviewmaketitle


\ifCLASSOPTIONcompsoc
\IEEEraisesectionheading{\section{Introduction}\label{sec:introduction}}
\else
\section{Introduction}
\label{sec:introduction}
\fi
% Computer Society journal (but not conference!) papers do something unusual
% with the very first section heading (almost always called "Introduction").
% They place it ABOVE the main text! IEEEtran.cls does not automatically do
% this for you, but you can achieve this effect with the provided
% \IEEEraisesectionheading{} command. Note the need to keep any \label that
% is to refer to the section immediately after \section in the above as
% \IEEEraisesectionheading puts \section within a raised box.




% The very first letter is a 2 line initial drop letter followed
% by the rest of the first word in caps (small caps for compsoc).
% 
% form to use if the first word consists of a single letter:
% \IEEEPARstart{A}{demo} file is ....
% 
% form to use if you need the single drop letter followed by
% normal text (unknown if ever used by the IEEE):
% \IEEEPARstart{A}{}demo file is ....
% 
% Some journals put the first two words in caps:
% \IEEEPARstart{T}{his demo} file is ....
% 
% Here we have the typical use of a "T" for an initial drop letter
% and "HIS" in caps to complete the first word.
\IEEEPARstart{M}{usic} preference has many dimensions, one of which is the emotional state of the listener \cite{Zwaag}. It is difficult to imagine listening to loud, high-energy music when one wishes to remain calm or focus on a task, as suggested by J.H. Janssen et al. \cite{Janssen}. Additionally, music can have a therapeutic effect on the listener. For instance, studies have taken this approach to using music to aid cancer patients and their families \cite{Bailey}. In our project, we are interested in the relationship between the listener's mood and music. We aim to identify the listener's emotional state by analyzing a text description, and then provide songs that can either match their emotional state or improve it.

For the first part of our project, we performed a survey to identify well-established techniques and the state of the art. We begin by researching ways to quantify emotion. There are different models used for identifying the specific emotional state which is closest to a song. Existing techniques on emotional analysis are based on music extract properties from a song like BPM, genre and timbre \cite{Kim}. After being extracted from the song, these features are used to quantify the mood or emotional context. For the following part of our project we do the inverse mapping: we identify ways to connect the listener's mood to a song. In \emph{Tune into your emotions} (Janssen et al. 2011) \cite{Janssen} it is described how biosignals are translated to emotions and music is retrieved to affect these emotions in a specific way. Our project is similar, however, our approach will be more business oriented and we will be gathering emotions in a more accessible way; from a text. In this phase, we will research how accurately we can give music suggestions based on emotional analysis of text fragments.

After researching these topics, we aim to produce a multimedia system that is able to identify the listener's emotions through text and map them to songs that match this mood. As suggested by Sauro and Lewis \cite{Sauro}, the main factor of a success for a system is if it can capture the user as the main actor.  We aim to achieve this by providing an intuitive user interface and being mindful of the user when constructing our algorithm. During this process, we found novel ways to use emoticons for this purpose. 

Lastly, this task presents valuable societal impact and business potential. The societal impact of recommendation systems is great for it allows for media discovery, allowing unheard artists to be heard. Additionally, with the identification of emotional content we bring a new dimension to recommendation systems in general that can better gauge the receptiveness of a user to media. This gives an opportunity to more advertisement aware platforms. Through this project, we hope to both explore the state of the art of multimedia technologies whilst producing a robust multimedia system.

\section{Survey}
In this section, we describe the prevalent approaches in Music Emotion Recognition for representing emotions. We distinct categorical and dimensional models. Categorical representations are for example the 66 adjectives arranged in 8 groups as identified by Hevner \cite{Hevner}; the 146 emotional terms specific for music mood rating proposed by Zentner et al. \cite{Zentner}; the 5 mood clusters from labels from the All Music Guide, as used in the MIREX competition \cite{Hu}; and PASAS \cite{Watson}.
\begin{figure}[h!]
	\centering
	\includegraphics[width=0.5\textwidth]{"Figure1_ValenceArousal"}
	\caption{The Valence-Arousal space, labeled by Russell's direct circular projection of adjectives. Source: \cite{Russel}}
	\label{fig:va}
\end{figure}

Another approach is to map emotions to a dimensional model. The most common dimensional model is the valence-arousal space, as proposed by Russell and Thayer \cite{Russel}. It is a 2D model, with valence on the x-axis and arousal on the y-axis. Valence tells something about the positivity of an emotion, so "happy" has a high valence, but the valence of "miserable" is extremely low. Arousal is about the excitement that is attached to an emotion, so "alarmed" has a way higher arousal value than “tired”.

In our system, we choose the valence-arousal model. With this model, we expect to be able to suggest songs of which the emotion fits relatively well to the emotion in the text that was inputted by the user. If we would choose a categorical model, every song in a category would have the same probability of being suggested - this might give the user the impression that the music was chosen randomly and not based on his emotion. Using the dimensional valence-arousal model, we can easily compare distances between the emotion in the text and in the music, and suggest the song to which the distance is smallest.

\section{Methods}

\subsection{Text Analysis}
For deciding what the sentiment of a text is, we map values of the text given by the user to a valence-arousal space. We tokenize the input and split the tokens into three groups: words, emoticons and punctuation. 

For the words, we used a dictionary that maps English words to corresponding valence-arousal values. We disregard the text's grammar, order and context and consider the text as a bag-of-words. This approach was described for the first time in ANEW (Bradley and Lang) \cite{Bradley}. We decided to use an updated dictionary, Warriner et al.\cite{Warriner}'s emotional ratings, which has some advantages and disadvantages. The ANEW-dictionary was created in 1999 and since language develops fast the valence and arousal values of specific words may have altered over time. The dictionary contains about 14.000 words whereas ANEW only contains roughly 1000 words. Having more words makes it more suitable for small text snippets, like tweets, as it is more likely to have matching words in the dictionary. It is a disadvantage on larger documents because it contains words with relatively neutral valence and arousal values. Because of this the longer the document the more it will go to the center of the vector space.

When the mean and standard deviation of both the valence and arousal are retrieved, we make a result vector of the text input with formula (\ref{eq:typev}). 
\begin{equation}
	v = \sum\limits_{i=1}^n\frac{v_{i,\mu}}{\sum (v_{\sigma}^{-1})v_{i,\sigma}}
	\label{eq:typev}
\end{equation}
This vector is calculated by averaging the vectors of the words weighted by the reverse of the standard deviation. This way the weight of a word gets bigger if the word has the a similar vector in most contexts. The result vector gets multiplied with the average of the standard deviations of its component vectors to preserve the right length.

\subsection{Emoticon Analysis}
However, we have expanded this approach to include punctuation and emoticons. This modification is done to improve the valence-arousal determination on a short text. Sentiment analysis on short text is challenging and does not present satisfactory results or has a limited categorization of emotions. As such,  the valence-arousal values for punctuation and the two kinds of emoticons were determined by the questionnaire (figure \ref{fig:eq}), and handed out to 50 people to fill in. We created two versions of the questionnaire and both versions contained half of the 50 most used emoticons. We processed the results and created a dictionary similar to the word dictionary from the previous section (figure \ref{fig:emoji}). 
\begin{figure}[h!]
	\centering
	\includegraphics[width=0.5\textwidth]{"EmoticonQuestionnaire"}
	\caption{Emoticon questionnaire, with a likert-scale from 1 to 5 for both valence and arousal.}
	\label{fig:eq}
\end{figure}
\begin{figure}[h!]
	\centering
	\includegraphics[width=0.5\textwidth]{"Emoji"}
	\caption{The emoticons mapped on the VA-space, with valence and arousal from -2 to 2.}
	\label{fig:emoji}
\end{figure}

To obtain the valence and arousal of a text input with emoticons and punctuation we first retrieved the separate vectors for the text, the emoticons and the punctuation by getting the values from the dictionary and combining them with formula (\ref{eq:resultv}). After we combined these 3 vectors by scaling them with specific weights and summing them.
\begin{equation}
v_{r} = \frac{w_{p}v_{p} + w_{e}v_{e} + w_{t}v_{t}}{w_{p}w_{e}w_{t}}
\label{eq:resultv}
\end{equation}

where $v = (0,0)$ and $w = 1$, when the input did not contain tokens of that type. The complete process is showed in (figure \ref{fig:ta})

\begin{figure}[h!]
	\centering
	\includegraphics[width=0.48\textwidth]{"TextAnalysisOverview"}
	\caption{Overview of the text analysis process.}
	\label{fig:ta}
\end{figure}




% needed in second column of first page if using \IEEEpubid
%\IEEEpubidadjcol

\subsubsection{Audio Analysis}

As a next step, we needed a database of songs, labeled with valence and arousal emotion values. In this section, we'll explain how this was accomplished.  Due to limited time and scope of this projected we sought pre-collected music metadata. Our requirements for this data were; a broad range of genres, audio features, and lyrics. The Million Song Dataset \cite{Thierry} was the best candidate and is the dataset used in our system. The biggest advantage is that this dataset is the largest one that is freely available at present. It consists of a million contemporary popular music tracks, so the music would be appealing to a great audience. Moreover, each song comes with audio features and metadata.

The Million Song Dataset consists of one million audio features and metadata belonging to contemporary popular music tracks, in HDF5-format. These are provided by The Echo Nest, a music intelligence company, which was recently (March 2014) bought by Spotify. A list of all fields of files available in the dataset is listed on \href{http://labrosa.ee.columbia.edu/millionsong/pages/field-list}{Million Song Database Files}. There is a lot of information available for each song (all kinds of metadata, corresponding ID's of this song in other datasets, etc), but we are mainly interested in audio features like loudness, mode and energy. The latter is described as "a perceptual measure of intensity and powerful activity released throughout the track". Even more audio features can be obtained by using the Echo Nest API. Additionally, Echo Nest offers a psychological label; valence. Using a combination of the valence and energy features, we can easily map each song to valence-arousal space. Note however that this only takes audio the information into account and does not use other data that could be relevant for the emotion associated with a song, for example, lyrics or tags. For our project, we used 4006 songs from the Million Song dataset that possess valence-arousal provided by EchoNest\footnote{Echo Nest gives little information about how this value is obtained. However they were able to inform us the valence values here are obtained solely by the audio processing of features such as pitch, mode and other acoustics drive features. General information about the features they provide can be obtained at \href{http://developer.echonest.com/acoustic-attributes.html}{Acoustic Attributes}}. You can observe the valence-arousal space of this data on figure \ref{fig:songva}.

\begin{figure}[h!]
	\centering
	\includegraphics[width=0.5\textwidth]{"SongVA"}
	\caption{Valence and arousal values for 4006 songs. These values were retrieved from the EchoNest.}
	\label{fig:songva}
\end{figure}

Additionally, the Million Song Dataset does not offer lyrics data. Fortunately, a connection to other datasets that do can easily be made, as the Million Song Dataset provides ID's of a song in those other datasets. Lyrics in bag-of-words format are provided by musiXmatch and are directly associated with 237.662 MSD tracks. Tags can be obtained from the Last.fm dataset - 505.216 of the songs from the Million Song Dataset have at least one tag. Audio data is not in the MSD for copyright reasons. 

Furthermore, in our Music Emotion Recognition component, we combine the emotion obtained from the audio data with the emotion obtained from the lyrics. Research by Yang et al [8] shows that this multimodal approach can greatly enhance audio-based classification algorithms: a relative improvement gain in classification accuracy of up to 21\% is observed when adding lyric features to the classification algorithm. However, the mapping here is solely to the four quadrants of valence-arousal space. In our project, we map these results to the entire valence-arousal space.

There are various possibilities for doing the fusion of audio and lyrics. Yang et al. \cite{Yang} mention three multimodal fusion methods:
\begin{itemize}
	\item{Early-fusion-by-feature-concatenation (EFFC): combine feature vectors of audio and lyrics to a single feature vector and make a classifier from this.}
	\item{Late-fusion-by-linear-combination (LFLC): train classifiers from audio and lyrics separately and combine their predictions afterwards in a linear way.}
	\item{Late-fusion-by-subtask-merging (LFSM): train audio and lyrics classifiers to classify valence and arousal separately and merge the result afterwards. So the difference is that the two dimensions go separated into the merging step.}
\end{itemize}
\begin{table}[]
	\centering
	\caption{Performance comparison of a number of multimodal fusion methods for four-class emotion classification, arousal classification and valence classification. Source: \cite{Yang}}
	\begin{tabular}{|c|p{1.5cm} p{1.5cm} p{1.2cm} p{1.3cm}|}
		\hline
		Methods & Number of Features & Four-Class Accuracy & Valence Arousal & Arousal Accuracy \\ \hline
		Audio-Only & 106 & 46.63\% & 61.15\% & 78.03\% \\
		Text-Only & 4000 & 40.01\% & 73.32\% & 61.95\% \\
		EFCC & 4106 & 52.48\% & 70.54\% & 77.06\% \\
		LFLC$_{0.5}$ & 106/4000 & 55.38\% & \textbf{74.83\%} & 77.88\%\\
		LFSM & 106/4000 & \textbf{57.06\%} & 73.32\% & \textbf{78.03\%}\\ 
		\hline
	\end{tabular}
	\label{tb:ya}
\end{table}
The researchers compared the performances of these approaches and found out that LFSM gave the best accuracy results, as can be seen in table \ref{tb:ya}. However, we see that the accuracy for valence is higher when using LFLC.

That is why we choose a combination of LFSM and LFSM in our system, which is possible as we do regression. We will calculate the valence as a linear combination of the valence from audio features and lyrics while we calculate arousal by using just the audio features:

\begin{equation}
Valence_{audio, lyrics} =  Valence_{audio} + (1 - \alpha)Valence_{lyrics}
\label{eq:valence}
\end{equation}
Formula (\ref{eq:valence}) shows how to determine the valence of a song, using the valence value from both audio and lyrics. $\alpha$ is a scalar value between 0 and 1.
\begin{equation}
Arousal_{audio, lyrics} =  Arousal_{audio}
\label{eq:arousal}
\end{equation}
Formula (\ref{eq:arousal}) shows how to determine the arousal of a song, using only the arousal value from audio.

Initially, we will choose $0.5$ as value for $\alpha$. In the experiments, we will experiment with other values as well.


\subsection{Emotion from lyrics}
	As explained previously, the lyrics we retrieved from musiXmatch will be considered as a bag-of-words. This limited our possibilities for the lyric emotion detection algorithm a little, as we could only use unigrams. We adapted the sentiment analysis algorithm we already created for the user input to lyrics. There are some differences: we do not interpret emoticons and interpunction in lyrics, as these do not occur in this domain. Also, we omitted the step of converting the text to the bag-of-words model, because the lyrics are already in this format. The motivation behind our approach is that in this manner we can best match the user's mood during sentiment analysis.
	
	\subsection{Reverse engineering of audio energy and arousal}
	The valence and arousal values we use from the audio are taken almost directly from the Echo Nest API. Unfortunately, little is provided in regards to how they obtained these values. However, we expect them to perform reasonably well (at least better than we could do in the scope of this project), but still we would like to know on which features their algorithm is based. If we know something about this, then it is easier to create our own algorithm so we would be less dependent on a company.
	
	That is why we extracted several audio features from the Million Song Dataset. We chose features that are relevant for Music Emotion Recognition: loudness, tempo, mode, mode confidence, key, key confidence and timbral features. The latter are MFCC-like features, which are 12-dimensional vectors with timbral information, calculated for each 30-second segment of a song. We calculate the mean and standard deviance for each element over all the segments, so we obtain a 24-dimensional vector for the timbre of a song. Using the valence and energy labels from the Echo's Nest as ground truth, we then trained a regressor on a part of the data (the training set) and tested its accuracy on the other part (the test set).
	
	For some songs, the mode confidence and/or key confidence is very low. In this case, we don't want our regressor to take mode and/or key into account for that song. That is why we replaced the mode value from songs with a mode confidence below 0.25 by the mean mode value. We replaced the key value from songs with a key confidence below 0.25 by the mean key value. 
	
	For performance evaluation, we used the R2 statistic, which is calculated as follows:
	
	\begin{equation}
	Valence_{audio, lyrics} =  Valence_{audio} + (1 - \alpha)Valence_{lyrics}
	\label{eq:statistic}
	\end{equation}
	
	We split the data into a training set of 2000 songs and a test set of 1992 songs. Then we tried some regression methods: multiple linear regression and support vector machines with a linear, polynomial and radial kernel. Each time, we trained a regressor for valence and arousal separately. We report the resulting R$^{2}$-values for the training and test set in table \ref{tb:rv}.
	
	\begin{table}[]
		\centering
		\caption{R$^{2}$-values for regressors on training and test set}
		\begin{tabular}{|c|p{1.5cm} p{1.5cm} p{1.5cm} p{1cm}|}
			\hline
			Regressor & Valence Train & Valence Test & Energy Train & Energy Test \\ \hline
			MLR & 0.47 & 0.46 & 0.73 & 0.74 \\
			SVM linear & 0.48 & 0.45 & 0.73 & 0.74 \\
			SVM polynomial & 0.64 & 0.27 & 0.77 & 0.45 \\
			SVM radial & 0.75 & 0.51 & 0.85 & 0.73 \\
			\hline
		\end{tabular}
		\label{tb:rv}
	\end{table}
	
	We can conclude that the SVM regressor with a radial kernel gives us the best model. However, the results are not that good so to get a better fit, we probably need to take other audio features into account. In future work, we could do more experiments with for example pitch features.
		
	\section{System description}	
	
In this, section we give a broad overview of our application as well as describe how our work can be reproduced. The description offered here is a user-centric description, so we first start with the user interaction with the system. The user main interaction is with a text box. In this text box, the user enters text about their current mood. 

Sentiment analysis is done on the text we obtained. As previously mentioned in our methods section, we use a dictionary approach to map each word entered to a valence and arousal value. These values are matched to the closest valence and arousal values present in the database. A song is then selected. Finally, Youtube is searched by using the song title and artist name. This video is then played to the user.

\begin{figure}[h!]
	\centering
	\includegraphics[width=0.5\textwidth]{"schema"}
	\caption{System overview of MARS.}
\end{figure}	
	
	\section{System Performance}
	In our experiments, we tested the system performance by asking university students to use the system and fill in a questionnaire (Please see appendix documents for questionnaire). We had 5 respondents. 
	
	We asked them how they felt before and after using the system, to get an indication of whether they liked it or not. In addition, we gave them five tasks: they had to write messages that were happy, sad, neutral, relaxed and angry. After each task, we asked them five questions. First, we asked whether they found it difficult to type the message, so we could validate if typing text is a convenient way for expressing emotions. Then we whether they liked the music, as this can have an effect on the user perception of the system performance. The next three questions were used to tweak parameters: we asked them whether they thought the mood of the music matched the mood of the text; whether they thought the music was too happy and whether they thought the music was too aroused. All these questions were asked using a 7-point Likert scale. Finally, there was a field for general remarks.
	
		\begin{table}[]
			\centering
			\caption{Final users comments. Please see appendix documents for the entire evaluation breakdown.}
			\begin{tabular}{|c|p{6cm}|}
				\hline
				& Q3 \\ \hline
				Person 1 & Nice system, I might use it :) \\
				Person 2 & Works pretty well! \\
				Person 3 & \\
				Person 4 & Usability issues \\
				Person 5 & The positive emotions work better than the negative \\
				\hline
			\end{tabular}
			\label{tb:comments}
		\end{table}
		
		Overall, the users of the system were impressed how the emotions described in the text were matched to a song. One user did indicate that there were some usability issues in regards to the interface of the website. Another indicated that sad emotions are not as well represented as happy.
	
	\section{Discussion}
	
	There are some limitations in our project. In this section, we discuss them and suggest solutions that can potentially mitigate these issues. One of the main factors we omitted in our project is music preference. As explained by Janssen et al \cite{Janssen}, music preference changes how the user is affected by music. The example given is that listeners familiar with high tempo music such as heavy metal may perceive Heavy Metal songs to be less arousing than someone not as familiar with the genre. Due to the scope and time, we were unable to capture this dimension in our system. However, given the plethora availability of music services online, this task is not unfeasible. This issue could have been mitigated by requesting user data from services such as Spotify. This way we could better determine what the listener listens to most often and their genre preferences. 
	
	An additional limitation with our approach is the data used, specifically, the valence-arousal values retrieved from Echo Nest. Little information is provided on how these values are produced. Nevertheless, there are reasons to believe that these values are not labels produced by humans but rather the result of a proprietary algorithm. Due to the time and scope of our project this is an issue that is difficult to mitigate. However, observation of samples of the dataset indicate that the values match what a listener would expect. We attempted to identify and reverse engineer the process using a  regressor model. Due to our limitations, however, it was not possible to achieve ground truth with the dataset provided. We strongly recommend further work to do this as it may improve results.
	
	Furthermore, while the valence-arousal space can capture a wide range of emotions, it can produce results that lack accuracy. For example, mad or upset are closely mapped but ideally the music recommended for these two states should differ significantly. Intuition indicates that there are two main causes for this behavior; closely defined values in the database and ill-defined emotions. The former issue may be able to be improved by scaling, and a further selection of songs to ensure songs in the database are well spread. The latter issue is more complex, however. This may be a limitation of the valence-arousal model. However, as mentioned previously, other sentiment models do not provide the same flexibility, therefore valence and arousal models are still better for a broader range of sentiment analysis.
	
	



% An example of a floating figure using the graphicx package.
% Note that \label must occur AFTER (or within) \caption.
% For figures, \caption should occur after the \includegraphics.
% Note that IEEEtran v1.7 and later has special internal code that
% is designed to preserve the operation of \label within \caption
% even when the captionsoff option is in effect. However, because
% of issues like this, it may be the safest practice to put all your
% \label just after \caption rather than within \caption{}.
%
% Reminder: the "draftcls" or "draftclsnofoot", not "draft", class
% option should be used if it is desired that the figures are to be
% displayed while in draft mode.
%
%\begin{figure}[!t]
%\centering
%\includegraphics[width=2.5in]{myfigure}
% where an .eps filename suffix will be assumed under latex, 
% and a .pdf suffix will be assumed for pdflatex; or what has been declared
% via \DeclareGraphicsExtensions.
%\caption{Simulation results for the network.}
%\label{fig_sim}
%\end{figure}

% Note that the IEEE typically puts floats only at the top, even when this
% results in a large percentage of a column being occupied by floats.
% However, the Computer Society has been known to put floats at the bottom.


% An example of a double column floating figure using two subfigures.
% (The subfig.sty package must be loaded for this to work.)
% The subfigure \label commands are set within each subfloat command,
% and the \label for the overall figure must come after \caption.
% \hfil is used as a separator to get equal spacing.
% Watch out that the combined width of all the subfigures on a 
% line do not exceed the text width or a line break will occur.
%
%\begin{figure*}[!t]
%\centering
%\subfloat[Case I]{\includegraphics[width=2.5in]{box}%
%\label{fig_first_case}}
%\hfil
%\subfloat[Case II]{\includegraphics[width=2.5in]{box}%
%\label{fig_second_case}}
%\caption{Simulation results for the network.}
%\label{fig_sim}
%\end{figure*}
%
% Note that often IEEE papers with subfigures do not employ subfigure
% captions (using the optional argument to \subfloat[]), but instead will
% reference/describe all of them (a), (b), etc., within the main caption.
% Be aware that for subfig.sty to generate the (a), (b), etc., subfigure
% labels, the optional argument to \subfloat must be present. If a
% subcaption is not desired, just leave its contents blank,
% e.g., \subfloat[].


% An example of a floating table. Note that, for IEEE style tables, the
% \caption command should come BEFORE the table and, given that table
% captions serve much like titles, are usually capitalized except for words
% such as a, an, and, as, at, but, by, for, in, nor, of, on, or, the, to
% and up, which are usually not capitalized unless they are the first or
% last word of the caption. Table text will default to \footnotesize as
% the IEEE normally uses this smaller font for tables.
% The \label must come after \caption as always.
%
%\begin{table}[!t]
%% increase table row spacing, adjust to taste
%\renewcommand{\arraystretch}{1.3}
% if using array.sty, it might be a good idea to tweak the value of
% \extrarowheight as needed to properly center the text within the cells
%\caption{An Example of a Table}
%\label{table_example}
%\centering
%% Some packages, such as MDW tools, offer better commands for making tables
%% than the plain LaTeX2e tabular which is used here.
%\begin{tabular}{|c||c|}
%\hline
%One & Two\\
%\hline
%Three & Four\\
%\hline
%\end{tabular}
%\end{table}


% Note that the IEEE does not put floats in the very first column
% - or typically anywhere on the first page for that matter. Also,
% in-text middle ("here") positioning is typically not used, but it
% is allowed and encouraged for Computer Society conferences (but
% not Computer Society journals). Most IEEE journals/conferences use
% top floats exclusively. 
% Note that, LaTeX2e, unlike IEEE journals/conferences, places
% footnotes above bottom floats. This can be corrected via the
% \fnbelowfloat command of the stfloats package.




% if have a single appendix:
%\appendix[Proof of the Zonklar Equations]
% or
%\appendix  % for no appendix heading
% do not use \section anymore after \appendix, only \section*
% is possibly needed

% use appendices with more than one appendix
% then use \section to start each appendix
% you must declare a \section before using any
% \subsection or using \label (\appendices by itself
% starts a section numbered zero.)
%


% you can choose not to have a title for an appendix
% if you want by leaving the argument blank




% trigger a \newpage just before the given reference
% number - used to balance the columns on the last page
% adjust value as needed - may need to be readjusted if
% the document is modified later
%\IEEEtriggeratref{8}
% The "triggered" command can be changed if desired:
%\IEEEtriggercmd{\enlargethispage{-5in}}

% references section

% can use a bibliography generated by BibTeX as a .bbl file
% BibTeX documentation can be easily obtained at:
% http://mirror.ctan.org/biblio/bibtex/contrib/doc/
% The IEEEtran BibTeX style support page is at:
% http://www.michaelshell.org/tex/ieeetran/bibtex/
%\bibliographystyle{IEEEtran}
% argument is your BibTeX string definitions and bibliography database(s)
%\bibliography{IEEEabrv,../bib/paper}
%
% <OR> manually copy in the resultant .bbl file
% set second argument of \begin to the number of references
% (used to reserve space for the reference number labels box)
\begin{thebibliography}{30}
  
  \bibitem{Zwaag}van der Zwaag, Marjolein D., Joyce HDM Westerink, and Egon L. van den Broek, "Emotional and psychophysiological responses to tempo, mode, and percussiveness." Musicae Scientiae 15.2 (2011): 250-269.
  
  \bibitem{Janssen}Janssen, Joris H., Egon L. van den Broek, and Joyce HDM Westerink, "Tune in to your emotions: a robust personalized affective music player." User Modeling and User-Adapted Interaction 22.3 (2012): 255-279.
  
  \bibitem{Bailey}Bailey, Lucanne Magill, "The use of songs in music therapy with cancer patients and their families." Music Therapy 4.1 (1984): 5-17.
  
  \bibitem{Kim}Kim, Youngmoo E., et al., "Music emotion recognition: A state of the art review." Proc. ISMIR (2010).
  
  \bibitem{Sauro}Sauro, Jeff, and James R. Lewis., "Quantifying the user experience: Practical statistics for user research." Elsevier (2012).

  \bibitem{Hevner}K. Hevner, "Experimental studies of the elements of expression in music." American Journal of Psychology (1936): vol. 48, no. 2, pp. 246–267.
     
  \bibitem{Zentner}M. Zentner, D. Grandjean, and K. R. Scherer, "Emotions evoked by the sound of music: Characterization, classification, and measurement." Emotion (2008), vol. 8, p. 494.
   
  \bibitem{Hu} X. Hu, J. Downie, C. Laurier, M. Bay, and A. Ehmann, "The 2007 MIREX audio mood classification task: Lessons learned." in Proc. of the Intl. Conf. on Music Information Retrieval, Philadelphia, PA (2008).
    
  \bibitem{Watson}D. Watson and L. Clark, "PANAS-X: Manual for the Positive and Negative Affect Schedule, expanded form ed." University of Iowa (1994).
  
  \bibitem{Russel}Russell, James A., "A circumplex model of affect." Journal of personality and social psychology 39.6 (1980): 1161.
  
  \bibitem{Bradley}Bradley, M.M., Lang, P.J., "Affective norms for English words (ANEW): Instruction manual and affective ratings". Technical Report C-1, The Center for Research in Psychophysiology, University of Florida (1999).
  
  \bibitem{Warriner}Amy Beth Warriner, Victor Kuperman, Marc Brysbaert, "Norms of valence, arousal, and dominance for 13,915 English lemmas." Psychonomic Society, Inc. (2013)
  
  \bibitem{Thierry}Thierry Bertin-Mahieux, Daniel P.W. Ellis, Brian Whitman, and Paul Lamere, "The Million Song Dataset." In Proceedings of the 12th International Society for Music Information Retrieval Conference (ISMIR 2011)
  
  \bibitem{Yang}Yang, Yi-Hsuan, and Homer H. Chen. "Music emotion recognition." CRC Press, (2011): Chapter 10
  


\end{thebibliography}

% biography section
% 
% If you have an EPS/PDF photo (graphicx package needed) extra braces are
% needed around the contents of the optional argument to biography to prevent
% the LaTeX parser from getting confused when it sees the complicated
% \includegraphics command within an optional argument. (You could create
% your own custom macro containing the \includegraphics command to make things
% simpler here.)
%\begin{IEEEbiography}[{\includegraphics[width=1in,height=1.25in,clip,keepaspectratio]{mshell}}]{Michael Shell}
% or if you just want to reserve a space for a photo:


% You can push biographies down or up by placing
% a \vfill before or after them. The appropriate
% use of \vfill depends on what kind of text is
% on the last page and whether or not the columns
% are being equalized.

%\vfill

% Can be used to pull up biographies so that the bottom of the last one
% is flush with the other column.
%\enlargethispage{-5in}



% that's all folks
\end{document}


